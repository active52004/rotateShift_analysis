\documentclass{article} 
\usepackage{amsthm}
\newtheorem{theorem}{Theorem}[section]
\newtheorem{corollary}[theorem]{Corollary}
\begin{document} 

\section{The Collision Probability ofY Set} 
\paragraph{Glossary} The symbol, abbreviation and marks needed in
expressing the shift stage and probability computation.
\subsection{Preliminaries} 
\paragraph{Rotate Shifting} 
This paragraph introduce the mechanism of block rotate shifting stage in CETD.  
\begin{itemize} 
	\item rotate shifting is like whirling a wheel.  
	\item the output blk of rotate shifting is determined by the input blk and
the shift control parameter(No. of bits shifted) 
	\item The value of input and output blks may not be 1-1 map.
\end{itemize} 
Unlike logical shifting and arithmetic shifting, rotate shifting
behaves like whirling a wheel. The empty position in a message blk shifted is
filled by the bits shifted out. Figure 1 express the concept of rotate shift. 

We can refer that the result of rotate shifting a message blk depends on the
value of blk and the bits shifted.  If two message blocks whose value are
identical are shifted with distinct R$_i$s, the result blocks may be
identical.That means when R$_i$ fixed,the mapping from message blocks to the
shifted result blocks is not injection. This case is expressed in Figure 2(a).
For two distinct message blocks, however, their result blocks may be identical
when their R$_i$s are distinct. This case is expressed in Figure 2(b).  When
the R$_i$s of two message blocks are identical, the equality of result blocks
is same as the equality of their input blocks.

\subsection{Rotate Shifting and Y Set Collision} 
\paragraph{Y Set Collision}
This paragraph should introduce some simple examples of Y Set Collision,
explain in detail that what kind of input pair can cause Y set collision.
\begin{itemize} 
	\item The blk-pair equality is maintained:(pattern and single
block collision) 
	\item Two distinctive blks produce an identical blk
pair(shifting base and the effect on Y set) 
	\item A special case, blks
containing internal pattern and derived from same base.  
\end{itemize}
\paragraph{Replay Attack} 
In the context of this paper, replay attack refers to the circumstances that the
adversary copy the data-tag message block pair from
memory at an old time point and try to pass the verification with this old copy
at a new time point. In the scenario of replay attack , the data part and tag
part of two message block pairs are identical respectively. 
The security of a
tag generation design is evaluated by the probabilty that two tags equal when
their relative data parts are identical. This probability is expressed as
Pr[Tag Collision] in this paper.  Figure x shows the status of each stage in
CETD under replay attack. In this section , we analyze the behaviour of old and
new Y set. (figure show two CETD scheme, representing old and new data) 
\subsubsection{Y Set Collision Under Replay Attack}
\paragraph{Single Block Collision}
As shown in Figure 2(a), two identical X blocks can result two identical Y blocks while their relative R$i$ are distinct. We found that the identical X block pair X$_a$=X$_b$ resulting identical Y blocks with distinct R$i$s have a common property, this property is expressed in Theorem 1.
\begin{theorem}
Assume the two identical X blocks have same number of bits N=2$^n$. The result blocks Y$_a$ and Y$_b$ from rotate shifting two identical X blocks X$_a$=X$_b$ with distinct R$_a$ and R$_b$, can be identical only when X$_a$ is formed by repeating a binary pattern P, which is a binary segment. The length of this pattern P can be expressed as:

	P\_L = 2$^p$, p$\in$[0,n-1]
\end{theorem}

\paragraph{The Collision of Block Set}
From Figure 2(b) we can see that two distinct X block X$_a$ and X$_b$ can result two identical Y blocks Y$_a$=Y$_b$ when the relative R$_a$ $\neq$ R$_b$. Assume there are two identical X block sets X\_A and X\_B and the number of elements in each set is M. X\_A[i] = X\_B[i] for all i$\in$ [1,m] and none of the element is formed with pattern. We found that the result Y sets Y\_A and Y\_B can be identical in set level.  These two X sets resulting two set level identical Y sets have a common property. This property is expressed in Theorem 2.

\begin{theorem}
Assume the two X block set X\_A and X\_B have same number of element and this number is M. X\_A[i]=X\_B[i] for all i$\in$[1,m] and none of the element has pattern discussed in Theorem 1.1. Given two set level distinct shifting parameter sets R\_A and R\_B, the result sets Y\_A and Y\_B can be identical in set level if and only if:
	
	All the elements in X\_A are formed by shifting a base block X$_{base}$. 
\end{theorem}
\begin{corollary}
	If each element in the X sets discussed in Theorem 1.2 is formed by binary pattern discussed in Theorem 1.1, the result set Y\_A and Y\_B can be identical in both set level and block level.
\end{corollary}
The proof of Theorem 1 and Theorem can be refered in appendix.
\subsection{The Y Set Collision in Each Case Of Input} 
As discussed in Section x, 
\paragraph{Probability of Y Set Collision} 
This section should compute the Y set collision probability
under replay attack for each input case.  
\begin{itemize} 
	\item The condition of X set pair under replay attack: Xa[i] = Xb[i] 
	\item The Probability of Y set pair collision for each case of X set pair.  
	\item A general case of X set pair and the related Y set collision probability 
\end{itemize} 
\appendix
\section{Proof of Theorem 1.1 and Theorem 1.2}
\subsection{Proof of Theorem 1}
\subsection{Proof of Theorem 2}
\section{}
\end{document}
