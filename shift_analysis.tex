\documentclass{article}
\begin{document}
\section{The Collision Probability of Y Set}
\subsection{Preliminaries}
\paragraph{Rotate Shifting}
This paragraph introduce the mechanism of block rotate shifting stage in CETD. 
\begin{itemize}
	\item rotate shifting is like whirling a wheel.
	\item the output blk of rotate shifting is determined by the input blk and the shift control parameter(No. of bits shifted)
	\item The value of input and output blks may not be 1-1 map.
\end{itemize}
\paragraph{The Concept of Rotate Shifting}
Unlike logical shifting and arithmetic shifting, rotate shifting behaves like whirling a wheel. The empty position in a message blk shifted is filled by the bits shifted out. Figure 1 express the concept of rotate shift. 
\paragraph{The Behaviour of Rotate Shifting}
We can refer that the result of rotate shifting a message blk depends on the value of blk and the bits shifted.  

\subsection{Rotate Shifting and Y Set Collision}
\paragraph{Y Set Collision}
This paragraph should introduce some simple examples of Y Set Collision 
\begin{itemize}
	\item The blk-pair equality is maintained:(pattern and single block collision)
	\item Two distinctive blks produce an identical blk pair(shifting base and the effect on Y set)
	\item A special case, blks containing internal pattern and derived from same base. 
\end{itemize}
\subsection{The Y Set Collision in Each Case Of Input}
\paragraph{Probability of Y Set Collision}
This section should compute the Y set collision probability under replay attack for each input case.
\begin{itemize}
	\item The condition of X set pair under replay attack: Xa[i] = Xb[i]
	\item The Probability of Y set pair collision for each case of X set pair.
	\item A general case of X set pair and the related Y set collision probability
\end{itemize}
\end{document}
