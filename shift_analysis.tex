\documentclass{article} 
\usepackage{amsthm}
\usepackage{amsmath}
\usepackage{graphicx}
\usepackage{subfigure}
%\usepackage{subfig}
\usepackage{verbatim}
\newtheorem{theorem}{Theorem}[section]
\newtheorem{corollary}[theorem]{Corollary}
\newtheorem{defination}{Definition}[section]
\begin{document} 
\section{The Collision Probability of Y Set}
\paragraph{Glossary} The symbol, abbreviation and marks needed in
expressing the shift stage and probability computation.

\paragraph{Replay Attack and Y Set Collision}
In replay attack, the adversary replace a data-tag pair on the memory with a
pair copied from the same address at an old time point. That means for the two
pairs at different time point, the two message block sets and related tags are
identical respectively, while the nonce N$_A$ and N$_B$ are randomly generated and
the equality is unpredictable if their generator is of high quality.  That means
the shifting bits parameter segment on the nonce R\_A and R\_B, are randomly
generated.

In this scenario, the probability of a successful attack can be expressed as the equation 1.1:
\begin{defination}
Pr[Successful Replay Attack] = Pr[Tag$_a$ = Tag$_b$ $\mid$ (D\_A = D\_B) \& (R\_A and R\_B are random)]
							 = Pr[Y\_A = Y\_B $\mid$ (D\_A = D\_B) \& (R\_A and R\_B are random)]
\end{defination} 

We can see that the collision of Y set will directly leads to the collision of tag and cause the succeed of replay attack. 
For easy understanding, we assume the shuffle stage does not work at first, which means D\_A = X\_A = D\_B = X\_B. Hence the Y set is the output of rotate shifting stage in CETD, we will analyze the properties of block rotate shifting and the cases of input sets that can result Y set collision. 

\begin{figure}[htbp]
 \centering
 \includegraphics[scale=0.4]{./diagrams/rotate_right.pdf}
 \caption{The Concept of Rotate Shifting(Right)}
 \label{fig:1 }
\end{figure}

\paragraph{Rotate Shifting Introduction} 
Unlike logical shifting and arithmetic shifting, rotate shifting
behaves like whirling a wheel. The empty position in a message blk shifted is
filled by the bits shifted out. Figure 1 express the concept of rotate shift.


We can refer that the result of rotate shifting a message block depends on the
value of blk and the bits shifted.  If two message blocks whose value are
identical are shifted with distinct R$_i$s, the result blocks may be
identical.That means when R$_i$ fixed,the mapping from message blocks to the
shifted result blocks is not injection. This case is expressed in Figure 2(a).

For two distinct message blocks, however, their result blocks may be identical
when their R$_i$s are distinct. This case is expressed in Figure 2(b).  
When the R$_i$s of two message blocks are identical, the equality of result blocks
is same as the equality of their input blocks.				

\begin{figure}
\centering
\subfigure[Same Block Shifted Different Bits]{
\begin{minipage}[b]{0.45\textwidth}
\includegraphics[width=1\textwidth]{./diagrams/r_s_2bits.pdf} \\
\includegraphics[width=1\textwidth]{./diagrams/r_s_6bits.pdf}
\end{minipage}
}
\subfigure[Different Blocks Shifted Different Bits]{
\begin{minipage}[b]{0.45\textwidth}
\includegraphics[width=1\textwidth]{./diagrams/r_d_4bits.pdf} \\
\includegraphics[width=1\textwidth]{./diagrams/r_d_2bits.pdf}
\end{minipage}
}
 \caption{The Examples of Y Block Collision}
 \label{fig:2 }
\end{figure}


\begin{figure}
\centering
\subfigure[Only Single Block Equality]{
 \label{fig:y_single_e} %% label for first subfigure
\includegraphics[width=.5\textwidth]{./diagrams/y_single_equality.pdf}
}
%\hspace{1in}
\subfigure[Only Set Level Equality]{
\label{fig:y_set_e} %% label for second subfigure
\includegraphics[width=.5\textwidth]{./diagrams/y_set_equality.pdf}
}
\caption{X Set Pairs with Only One Type of Y Equality}
 \label{fig:y_e_single} %% label for entire figure
\end{figure}

\begin{figure}
\centering
\subfigure[Single Block Equality Case]{
 \label{fig:y_both_single} %% label for first subfigure
\includegraphics[width=.5\textwidth]{./diagrams/y_both_single.pdf}
}
%\hspace{1in}
\subfigure[Set Level Equality Case]{
\label{fig:y_both_set} %% label for second subfigure
\includegraphics[width=.5\textwidth]{./diagrams/y_both_set.pdf}
}
\caption{A X Set Pair with Two Types of Y Equality}
 \label{fig:y_both} %% label for entire figure
\end{figure}

\subsection{Rotate Shifting and Y Set Collision} 
Assume the shuffle stage does not work, then D\_A= X\_A=D\_B=X\_B,which means the following properties exist in X\_A and X\_B:
\begin{itemize}
	\item X\_A[i] = X\_B[i] for all i$\in$[1,M], M is the number of elements
	\item The equality of elements in a X set is uncertain.
\end{itemize}
All the analysis in this section is based on the assumption that shuffle stage does not work.
If these two X sets generate two identical Y sets with distinct R\_A and R\_B, then the two Y sets contains  at least one of the following properties:
\begin{itemize}
	\item Single Block Equality: Y\_A[i] = Y\_B[i] for all i$\in$[1,M] where M is the number of element
	\item Set Level Equality: Any element in Y\_A, marked as Y\_A[i], is identical to an element in Y\_B, marked as Y\_B[j] and i$\neq$j.
\end{itemize}
Figure 3 and Figure 4 express the examples of X set pair that lead to idential Y set pair with distinct R set pair. In this paper, we analyze the cases of the input set pair (X\_A and X\_B) of shifting stage in CETD under replay attack that can lead to Y set pair collision.



\subsubsection{Case of X Sets Resulting Y Set Collision}
\paragraph{Single Block Equality Case}
As shown in Figure 2(a), two identical X blocks can result two identical Y blocks while their relative shifting bit parameter R[i] and R[j] are distinct. 
We found that the identical X block pair X\_A[i]=X\_B[i] resulting identical Y blocks with distinct R\_A[i] and R\_B[i] have a common property, this property is expressed in Theorem 1.

\begin{theorem}
Assume X\_A[i] and X\_B[i] are two identical block from two set X\_A and X\_B with same index i. X\_A[i] and X\_B[i] have same number of bits N=2$^n$. The result of rotate shifting X\_A[i] and X\_B[i] with distinct shifting bit parameter R\_A[i] and R\_B[i] are marked as Y\_A[i] and Y\_B[i]. Then Y\_A[i] and Y\_B[i] can be identical only when X\_A[i] and X\_B[i] are formed by repeating a binary pattern P, which is a binary segment. The length of this pattern P can be expressed as:

	P\_L = 2$^p$, p$\in$[0,n-1]
\end{theorem}

Assume the two X sets X\_A and X\_B are identical and can  form single block equality. Base on Theorem 1.1, we can conclude the corollary of the relationship between X set pair and Y set pair for single block case. 
\begin{corollary}
Assume there is M elements in X\_A and X\_B. If there is at least one i$\in$[1,M], X\_A[i] = X\_B[i] is formed by repeating a pattern P$_i$ whose length is P\_L$_i$ = 2$^pi$, then Y\_A and Y\_B can be single block identical with some two distinct shifting bits parameter sets R\_A and R\_B.
\end{corollary} 

\paragraph{Set Level Equality Case}
From Figure 2(b) we can see that two distinct X block X\_A[i] and X\_B[j] can result two identical Y blocks Y\_A[i]=Y\_B[j] when the relative R\_A[i] $\neq$ R\_B[j]. 

Assume the element in a set with distinct indexes are distinct. We found that two distinct blocks X\_A[i] and X\_B[j] resulting identical Y blocks with distinct R\_A[i] and R\_B[i] have a common property. This property is expressed in Theorem 2.

\begin{theorem}
Assume X\_A[i] and X\_B[j] are two distinct blocks from two set X\_A and X\_B and i$\neq$j. X\_A[i] and X\_B[i] have same number of bits N=2$^n$. 
The result of rotate shifting X\_A[i] and X\_B[j] with distinct shifting bit parameter R\_A[i] and R\_B[i] are marked as Y\_A[i] and Y\_B[i]. 

Then Y\_A[i] and Y\_B[i] can be identical only when X\_A[i] can be rotate shifted to X\_B[i]
\end{theorem}

Assume the two X sets X\_A and X\_B are identical and can form set level equality. Base on Theorem 1.2, we can conclude the corollary of the relationship between X set pair and Y set pair for set level equality. 
\begin{corollary}
Assume there are M elements in X\_A and X\_B. If there is at least two pairs of data, marked as (X\_A[i],X\_B[j]) and (X\_A[j]),X\_B[i]) where i,j$\in$[1,M] and i$\neq$j, that X\_A[i] can be formed by shifting X\_A[j], then Y\_A and Y\_B can be set level identical with some two distinct shifting bits parameter sets R\_A and R\_B.
\end{corollary} 
\paragraph{An Intersection Case}
Base on Corollary 1.3 and Corollary 1.4, we can draw the following corollary for the X set pairs that can form both set level and single block equality:
\begin{corollary}
Assume there is M elements in X\_A and X\_B. If there are at least two pairs of data, marked as (X\_A[i],X\_B[j]) and (X\_A[j]),X\_B[i]) where i,j$\in$[1,M] and i$\neq$j, that X\_A[i] can be formed by shifting X\_A[j] and X\_A[i] = X\_B[i] or X\_A[j]=X\_B[j] is formed by repeating a pattern P$_i$ whose length is P\_L$_i$ = 2$^pi$ , then Y\_A and Y\_B can be either set level or single block identical with some two distinct shifting bits parameter sets R\_A and R\_B.
\end{corollary}
The proof of Theorem 1.1 and Theorem 1.2 and Corollary 1.2 to 1.5 can be refered in appendix.


\subsection{The Probability of Y Set Collision} 
Hence the shift bit parameter set R\_A and R\_B are randomly generated, the probability of Y set collision for each input case is determined by the number of specific combinations of R\_A and R\_B. The computation of Pr[Y\_A = Y\_B] is based on this idea.

\subsubsection{Single Block Equality Case}
\paragraph{The Probability of Two Identical Y Block}
As discussed in above section, if each element in the X set is formed by a pattern P with pattern lentgh P$_l$= 2$^p$ and none of the element can be formed by rotate shifting the left elements, the result Y sets Y\_A and Y\_B have probability of Single Block Equality(SBE). This probability can be expressed as the following way:
\begin{defination}
Pr[SBE] = $$\prod_{i=1}^M Pr[Y\_A[i] = Y\_B[i]]$$ M is the No. of elements in a set
\end{defination} 
Assume the shuffle stage of CETD does not work. That means D\_A[i]=X\_A[i]=X\_B[i]=D\_B[i] for all i$\in$[1,M]. Then Pr[Y\_A[i] = Y\_B[i]] can be expressed as Pr[Y\_A[i] = Y\_B[i] | X\_A[i] = X\_B[i] \& R\_A and R\_B are randomly generated]. We use Pr[Y block collision] to express this probability. 

If R\_A[i] = R\_B[i] for all i$\in$[1,M], then Pr[Y\_A[i] = Y\_B[i]] = 1. When R\_A and R\_B is distinct, then at least one R\_A[i]-R\_A[i] pair int two R sets is distinct. For a distinct R block pair, their related Y block pair can be identical when the related X pair is formed by pattern. The probability is expressed in Theorem 1.3:
\begin{theorem}
If the No. of bits of each block in X\_A[i]-X\_B[i] pair is N=2$^n$, the pattern length P$_l$=2$^p$ where p$\in$[0,n-1]. The pattern contains no internal sub-pattern, then Pr[Y block collision] = 1/2$^p$
\end{theorem}
If each X set contains M elements, Pr[SBE] = (1/2$^p$)$^M$.
The proof of theorem can be referred in appendix.

\subsubsection{Set Level Equality Case}
\paragraph{What to say in this section}
\begin{itemize}
	\item In general condition, Y\_A and Y\_B are multisets. If Y\_A and Y\_B are identical in set level, then Y\_B is a permutation of Y\_A.
	\item how to make two identical Y block with two distinct X block and R block: if there is no pattern, for each value of block, the responding Y block is definite. 1 r map to 1 Y value
	\item How to represent the expression of Pr[set level collision $\mid$ X collision \& R distinct]
	\item If Y\_A is identical to Y\_B in set level, element in Y\_B can be regarded as a multiset permutation of set Y\_A.
\end{itemize}

If an element in an X set, marked as X[i], can be formed by rotate shifting another element in X, marked as X[i] where i$\neq$j, then we call these two block "same base element". Two same base elements can result two identical Y blocks with distinct shifting bit R[i]$\neq$R[j]. 

When none of an element in a X set contains a pattern, we call this X set a no-pattern set. Assume X\_A and X\_B are two identical no-pattern sets. It is impossible to two distinct R sets on X\_A and X\_B to form two single-block-equality Y sets . However, if there is some elements in X\_A set are same base elemnts, Y\_A and Y\_B can have Set Level Equality(SLE). 

Assume all elements in X\_A set are same base elments. 



\subsubsection{The Intersection Case}
\paragraph{What to say in this section}
\begin{itemize}
	\item If there is pattern in the block, for each element, the mapping between r and Y is not 1-to-1.
	\item Compute set level equality first, then solve the severl r to 1 Y problem
	\item express the Probability with combinatorics
\end{itemize}

\subsection{A General Condition of X Set and Related Y Set Collision}
\paragraph{what to say in this section}


\appendix
\section{Proof of Theorem 1.1 and Theorem 1.2}
\subsection{Proof of Theorem 1}
\subsection{Proof of Theorem 2}
\section{Proof of Theorem 1.3}
\section{Proof of Theorem 1.4 and 1.5}
\subsection{Proof of Theorem 1.4}
\subsection{Proof of Theorem 1.5}
\end{document}
